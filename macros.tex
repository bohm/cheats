\pdfoutput=1
\pdfpkresolution=600

\input ucwmac.tex
\input ucw-ofs.tex
\input color

\language\czech

\landscape

%**start of header
\newcount\columnsperpage

\columnsperpage=3

\def\versionnumber{0.01}  % Version of this reference card
\def\year{2012}
\def\month{January}
\def\version{\month\ \year\ v\versionnumber}


\def\shortcopyrightnotice{
}

\def\copyrightnotice{
}

% make \bye not \outer so that the \def\bye in the \else clause below
% can be scanned without complaint.
\def\bye{\par\vfill\supereject\end}

\newdimen\intercolumnskip
\newbox\columna
\newbox\columnb

\def\ncolumns{\the\columnsperpage}

\message{[\ncolumns\space
   column\if 1\ncolumns\else s\fi\space per page]}

\def\scaledmag#1{ scaled \magstep #1}

% This multi-way format was designed by Stephen Gildea
% October 1986.

% Modified for Czech fonts.
\if 1\ncolumns
   \hsize 4in
   \vsize 10in
   \voffset -.7in
   \font\titlefont=\fontname\tenbf \scaledmag3
   \font\headingfont=\fontname\tenbf \scaledmag2
   \font\smallfont=\fontname\sevenrm
   \font\smallsy=\fontname\sevensy

   \footline{\hss\folio}
   \def\makefootline{\baselineskip10pt\hsize6.5in\line{\the\footline}}
\else
   \hsize 3.56in
   \vsize 7.95in
   \hoffset -.75in
   \voffset -.745in
   \font\titlefont=csbx10 \scaledmag2
   \font\headingfont=csbx10 \scaledmag1
   \font\smallfont=csr6
   \font\smallsy=cssy6
   \font\eightrm=csr8
   \font\eighti=csmi8
   \font\eightsy=cssy8
   \font\eightbf=csbx8
   \font\eighttt=cstt8
   \font\eightit=csti8
   \font\eightsl=cssl8
   \textfont0=\eightrm
   \textfont1=\eighti
   \textfont2=\eightsy
   \def\rm{\eightrm}
   \def\bf{\eightbf}
   \def\tt{\eighttt}
   \def\it{\eightit}
   \def\sl{\eightsl}
   \normalbaselineskip=.8\normalbaselineskip
   \normallineskip=.8\normallineskip
   \normallineskiplimit=.8\normallineskiplimit
   \normalbaselines\rm          %make definitions take effect

   \if 2\ncolumns
     \let\maxcolumn=b
     \footline{\hss\rm\folio\hss}
     \def\makefootline{\vskip 2in \hsize=6.86in\line{\the\footline}}
   \else \if 3\ncolumns
     \let\maxcolumn=c
     \nopagenumbers
   \else
     \errhelp{You must set \columnsperpage equal to 1, 2, or 3.}
     \errmessage{Illegal number of columns per page}
   \fi\fi

   % \intercolumnskip=.46in
   \intercolumnskip=.20in
   \def\abc{a}
   \output={%
       % This next line is useful when designing the layout.
       %\immediate\write16{Column \folio\abc\space starts with \firstmark}
       \if \maxcolumn\abc \multicolumnformat \global\def\abc{a}
       \else\if a\abc
        \global\setbox\columna\columnbox \global\def\abc{b}
         %% in case we never use \columnb (two-column mode)
         \global\setbox\columnb\hbox to -\intercolumnskip{}
       \else
        \global\setbox\columnb\columnbox \global\def\abc{c}\fi\fi}
   \def\multicolumnformat{\shipout\vbox{\makeheadline
       \hbox{\box\columna\hskip\intercolumnskip
         \box\columnb\hskip\intercolumnskip\columnbox}
       \makefootline}\advancepageno}
   \def\columnbox{\leftline{\pagebody}}

   \def\bye{\par\vfill\supereject
     \if a\abc \else\null\vfill\eject\fi
     \if a\abc \else\null\vfill\eject\fi
     \end}
\fi

% ***** Verbatim typesetting *****

% Verbatim typesetting is done by
%    \verbatim"stuff to verbatim typeset"
% Any character can be used in place of ".
% E.g. \verbatim?stuff? or \verbatim|stuff|.  Cf. TeXbook pp.380-382

\def\uncatcodespecials{\def\do##1{\catcode`##1=12}\dospecials}
\def\setupverbatim{\tt%
\def\par{\leavevmode\endgraf}\catcode`\`=\active%
\obeylines\uncatcodespecials\obeyspaces}
\def\verbatim{\begingroup\setupverbatim\doverbatim}
\def\doverbatim#1{\def\next##1#1{##1\endgroup}\next}

\def\\{\verbatim}
\def\ds{\displaystyle}
\def\SPC{\quad} % space between symbol and command

\parindent 0pt
\parskip 1ex plus .5ex minus .5ex

\def\small{\smallfont\textfont2=\smallsy\baselineskip=.8\baselineskip}

\outer\def\newcolumn{\vfill\eject}

\outer\def\title#1{{\titlefont\centerline{#1}}\vskip 1ex plus .5ex minus.5ex}

%\outer\def\section#1{\par\filbreak
%  \vskip 1ex plus 2ex minus 2ex {\headingfont #1}\mark{#1}%
%  \vskip 1ex plus 1ex minus .5ex}

\outer\def\section#1{\par\filbreak
   \vskip .75ex plus 1ex minus 2ex {\headingfont #1}\mark{#1}%
   \vskip .5ex plus .5ex minus .5ex}
\outer\def\subsection#1{\par\filbreak
   \vskip .75ex plus 1ex minus 2ex {\bf #1}\mark{#1}%
   \vskip .5ex plus .5ex minus .5ex}
% looks like a section, but does not use the splitting algorithm
\outer\def\fakesection#1{\par
   \vskip .75ex plus 1ex minus 2ex {\headingfont #1}%
   \vskip .5ex plus .5ex minus .5ex}
\def\paralign{\vskip\parskip\halign}

%\def\<#1>{$\langle${\rm #1}$\rangle$}

\def\begintext{\par\leavevmode\begingroup\parskip0pt\rm}
\def\endtext{\endgroup}

% smaller indenting of itemize lists for a compact format
\itemindent=0.15in
\itemnarrow=0in

% math-related definitions (compressed than usual):

\def\dfncolor{\color{red}}
\def\prfcolor{\color{green}}
\def\thmcolor{\color{blue}}
\def\lemcolor{\color[rgb]{1,0,1}}
\def\obscolor{\color{black}}

\def\dfn#1{{\bf \begingroup\dfncolor D\ifx#1\empty\else(\endgroup{\I #1}\begingroup\dfncolor)\fi:\endgroup}}
\def\prf#1{{\bf \begingroup\prfcolor P\ifx#1\empty\else(\endgroup{\I #1}\begingroup\prfcolor)\fi:\endgroup}}
\def\thm#1{{\bf \begingroup\thmcolor T\ifx#1\empty\else(\endgroup{\I #1}\begingroup\thmcolor)\fi:\endgroup}}
\def\lem#1{{\bf \begingroup\lemcolor L\ifx#1\empty\else(\endgroup{\I #1}\begingroup\lemcolor)\fi:\endgroup}}
\def\obs#1{{\bf \begingroup\obscolor O\ifx#1\empty\else(\endgroup{\I #1}\begingroup\obscolor)\fi:\endgroup}}
\def\res#1{{\bf \begingroup\obscolor R\ifx#1\empty\else(\endgroup{\I #1}\begingroup\obscolor)\fi:\endgroup}}

\def\st{{\rm\ t.ž.\ }}
\def\iff{\leftrightarrow}
\def\then{\rightarrow}
\def\rng{{\rm Rng}}
\def\dom{{\rm Dom}}
% my compact notation for $x_1, x_2, \dots, x_n$.
\def\mls#1#2{#1_{[1,#2]}}

% enumerability-related macros
\def\da{\downarrow}
\def\ua{\uparrow}
\def\daeq{\downarrow=}
\def\daneq{\downarrow\neq}
